\documentclass[10pt,twocolumn,letterpaper]{article}

\usepackage{iccv}
\usepackage{times}
\usepackage{epsfig}
\usepackage{graphicx}
\usepackage{amsmath}
\usepackage{amssymb}

% Include other packages here, before hyperref.

% If you comment hyperref and then uncomment it, you should delete
% egpaper.aux before re-running latex.  (Or just hit 'q' on the first latex
% run, let it finish, and you should be clear).
\usepackage[pagebackref=true,breaklinks=true,letterpaper=true,colorlinks,bookmarks=false]{hyperref}

\iccvfinalcopy % *** Uncomment this line for the final submission
\def\iccvPaperID{****} % *** Enter the ICCV Paper ID here
\def\httilde{\mbox{\tt\raisebox{-.5ex}{\symbol{126}}}}

% Pages are numbered in submission mode, and unnumbered in camera-ready
\ificcvfinal\pagestyle{empty}\fi
\begin{document}

%%%%%%%%% TITLE
\title{Application of Conditional Random Field in Image Salient Object Detection\\ with Local, Regional and Global Feature Extraction}

\author{Jimmy Lin\\
Australian National University\\
Canberra, Australia\\
{\tt\small \url{linxin@gmail.com}}
% For a paper whose authors are all at the same institution,
% omit the following lines up until the closing ``}''.
% Additional authors and addresses can be added with ``\and'',
% just like the second author.
% To save space, use either the email address or home page, not both
\and
Chris-Chau-Long\\
Australian National University\\
Canberra, Australia\\
{\small\url{}}
}

\maketitle
% \thispagestyle{empty}

%%%%%%%%% ABSTRACT
\begin{abstract}
    Making use of OpenCV, DARWIN and MSRA datasets, we detect the saliency of by extracting local, 
    regional and global saliency feature, 
    then combine those features with pre-fitted weights derived by logistic regression.
    On top of that, the conditional random field framework is constructed to capture the spatial continuity of the saliency.
    The importance ratio between the combined unary and pairwise term is determined by cross validation.
    Based on the binary mask inferred in Conditional Random Field, we ultimately apply winner-take-all algorithm to 
    output one boxing rectangle to label the detected salient object, by which the performance of our 
    approach is evaluated. 
    %%- respectively multiscale contrast, center surround histogram, and color spatial distribution. 
\end{abstract}

%% Introduction & Related Works (1 page)
\section{Introduction}

%% Formulation (1 page)
\section{Formulation}

%% Feature Extraction (2 pages)
\section{Feature Extraction}

%% Learning (0.5 page)
\section{Learning}

%% CRF Inference (0.5 page)
\section{CRF Inference}

%% Pixel Level result presentation (1 page)
\section{Result Presentation}

%% Result evaluation in rectangle level (1 page)
\section{Evaluation}

%% Discussion (existing flaws and possible improvements) (0.5 page)
\section{Discussion}
\subsection{Current Drawbacks}
\subsection{Possible Improvement}

%% acknowlegement and reference (0.5 page)
\begin{thebibliography}{99} \fontsize{9pt}{50} \setlength{\itemsep}{-0.5pt} 
    \bibitem 1 Liu, Tie, et al. "Learning to detect a salient object." 
        \textit{Computer Vision and Pattern Recognition, 2007. CVPR07. IEEE Conference on. IEEE, 2007}.

    \bibitem 2 Liu, Tie, et al. "Learning to detect a salient object." 
        \textit{Pattern Analysis and Machine Intelligence, IEEE Transactions on 33.2 (2011): 353-367}. 

    \bibitem 3 Itti, Laurent, Christof Koch, and Ernst Niebur. "A model of saliency-based visual attention for rapid scene analysis."
        \textit{Pattern Analysis and Machine Intelligence, IEEE Transactions on 20.11 (1998): 1254-1259}.

    \bibitem 4 Ma, Yu-Fei, and Hong-Jiang Zhang. "Contrast-based image attention analysis by using fuzzy growing."
        \textit{ Proceedings of the eleventh ACM international conference on Multimedia. ACM, 2003}. 

    \bibitem 5 Gould, Stephen. "DARWIN: A Framework for Machine Learning and Computer Vision Research and Development." 
        \textit{Journal of Machine Learning Research 13 (2012): 3533-3537}

\end{thebibliography}
%-------------------------------------------------------------------------

\end{document}

